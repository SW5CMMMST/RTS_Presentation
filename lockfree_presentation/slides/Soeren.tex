\begin{frame}[fragile, allowframebreaks]{What are lock-free datastructures?}
	\begin{block}{Compared to locked-datastructures}
		\begin{itemize}
			\item Datastructure using a standard \textcolor{ReneOrange}{read}.
			\item \textcolor{ReneOrange}{Updating} is another story.
			\item Atomic \textcolor{ReneOrange}{Read-Modify-Write} Operation
			\item Help from hardware
		\end{itemize}
	\end{block}
	\begin{block}{How does update work?}
		\begin{itemize}
			\item Obtain the pointer
			\item Make a copy of the data from the pointer
			\item Update the copied data
			\item \textcolor{ReneOrange}{Compare-and-Swap} the pointers
		\end{itemize}
	\end{block}

\begin{block}{Compare-and-Swap}
\begin{lstlisting}[style=customc]
bool CAS(int* addr, int expected, int value) {
	if (*addr == expected) {
	  *addr = value;
	  return true;
	}
	return false;
} 
\end{lstlisting}
\end{block}
\end{frame}

\begin{frame}{Implications of being lock-free}
	\begin{block}{What does this mean for the multi-threaded programs?}
		\begin{itemize}
			\item Reading can always be done, no thread or process have a \textcolor{ReneOrange}{lock} on it.
			\item One thread or process will always progress
			\item Can give an increase in performance when mostly reading
			\item If many threads or processes are trying to update, it might be very slow
			\item \textcolor{ReneOrange}{Starvation} can still occur
		\end{itemize}
	\end{block}
\end{frame}