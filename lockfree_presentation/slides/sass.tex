\begin{frame}[fragile,allowframebreaks]{Implementing lock-free datastructures}
\begin{block}{Simple Stack}
\begin{lstlisting}[style=customc]
class Node {
    Node * next;
    int data;
};
Node * head; //Stable stack head
\end{lstlisting}
\end{block} 

\begin{block}{Lock-free push}
\begin{lstlisting}[style=customc]
void push(int t) {
    Node* node = new Node(t);
    do {
        node->next = head;
    } while (!cas(&head, node, node->next));
}
\end{lstlisting}
\end{block}

\begin{block}{Lock-free pop}
\begin{lstlisting}[style=customc]
bool pop(int& t) {
    Node* current = head;
    while(current) {
        if(cas(&head, current->next, current)) {
            t = current->data;
            return true;
        }
        current = head;
    }
    return false;
}
\end{lstlisting}
\end{block}

\end{frame}

\begin{frame}[allowframebreaks]{When to use lock-free programming}
    \begin{block}{Cons}
    \begin{itemize}
        \item Implementation is tricky and difficult
        \item ABA Problem
        \item Lock-free is not wait-free
        \item No guarentee that a thread will terminate (it may wait forever)
        \item Memmory barriers are needed
    \end{itemize}
    \end{block}

    \begin{block}{Pros}
    \begin{itemize}
        \item Can improve speed (e.g. in liked lists where long traversals can occur)
        \item Easily scalable to an arbitrary number of threads
        \item Can prevent priority inversion
        \item \textcolor{ReneOrange}{No deadlocks!}
        \item Useful when lock are not available for shared data (e.g. interrupt handlers)
    \end{itemize}
    \end{block}
    
    \begin{exampleblock}{Conclusion}
        Should only be used as a last resort, but can be very powerfull when implemented properly
    \end{exampleblock}
\end{frame}
